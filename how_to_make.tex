%Set up the document and load packages

\documentclass{article}
\linespread{1.3}

\usepackage{arxiv}          % the arxive format
\usepackage[utf8]{inputenc} % allow utf-8 input
\usepackage[T1]{fontenc}    % use 8-bit T1 fonts
\usepackage{hyperref}       % hyperlinks
\usepackage{url}            % simple URL typesetting
\usepackage{booktabs}       % professional-quality tables
\usepackage{amsfonts}       % blackboard math symbols
\usepackage{nicefrac}       % compact symbols for 1/2, etc.
\usepackage{microtype}      % microtypography
\usepackage{hyperref}       % writing URLs
\usepackage{parskip}
\setlength{\parindent}{15pt}

\title{How to use {\sf Make} to automatically update projects dependent upon {\sf R} and \LaTeX}


\author{
    Joshua G. Harrison\\
    1000 E. University Ave.\\
    Department of Botany, 3165\\
    University of Wyoming\\
    Laramie, WY 82071, USA\\
  %% \AND
  %% Coauthor \\
  %% Affiliation \\
  %% Address \\
  %% \texttt{email} \\
  %% \And
  %% Coauthor \\
  %% Affiliation \\
  %% Address \\
  %% \texttt{email} \\
  %% \And
  %% Coauthor \\
  %% Affiliation \\
  %% Address \\
  %% \texttt{email} \\
}

\begin{document}
\maketitle

\section{Why use {\sf Make}?}

Inevitably, over the course of a project one finds themselves updating data and scripts many times. Some scripts depend upon other scripts, and updating any of them, or the underlying data, requires re-executing all scripts in order. One could make a {\sf bash} script to do this, but {\sf Make} offers several advantages over a home-cooked {\sf bash} script. First, {\sf Make} allows you to specify dependencies among your files which can expedite re-execution. An example of a dependency structure is: script x depends upon data y. When one runs {\sf Make} it will search specified files, find any that updated, and execute everything that depends on those files. When you have a large project, with a complex dependency structure this could provide large speed gains over {\sf bash}.  
In this document, I will provide an example of how to use {\sf Make} to sync updates in ones R scripts or data to a \LaTeX\ manuscript. 

Setting up a Makefile for your project will take a few minutes, but this time investment will be repaid on any but the simplest of projects.

\subsection{A brief history of {\sf Make} }

According to the all-knowing Wikipedia, {\sf Make} was invented by Stuart Feldman at Bell labs way back in 1976. Here is a quote from Feldman regarding the genesis of the program (swiped from Wikipedia; give them \$5 this year):

    \begin{quote}
        Make originated with a visit from Steve Johnson (author of yacc, etc.), storming into my office, cursing the Fates that had caused him to waste a morning debugging a correct program (bug had been fixed, file hadn't been compiled, cc *.o was therefore unaffected). As I had spent a part of the previous evening coping with the same disaster on a project I was working on, the idea of a tool to solve it came up. It began with an elaborate idea of a dependency analyzer, boiled down to something much simpler, and turned into Make that weekend. Use of tools that were still wet was part of the culture. Makefiles were text files, not magically encoded binaries, because that was the Unix ethos: printable, debuggable, understandable stuff.
        
        --Stuart Feldman, The Art of Unix Programming, Eric S. Raymond 2003
    \end{quote}

Since the 70s {\sf Make} has come standard with {\sf Unix} distributions. It is often used to compile software and update parts of software that rely on things that have changed. Thus, you can just compile the necessary parts of complicated software (like an operating system) without recompiling the whole thing and wasting time.

\section{The tutorial}
\subsubsection{Prerequisites}

You will need to have {\sf git}, {\sf R}, a {\sf TeX} distribution with \LaTeX, and {\sf Make} installed. On an Apple computer you can install {\sf Make} with the developer tools (to figure out how to do this Google ``installing Xcode tools mac''). To install {\sf Make} on Windows see this page for a place to start, \url{http://stat545.com/automation02_windows.html}. 

If you wish to render pdfs from \LaTeX\ files outside of {\sf Overleaf} then you will want to also install pdfTeX (or something similar). If you are on an Apple, then you can install {\sf mactex}, which will come with {\sf pdflatex}. Note, that you may need to add the {\sf TeX} library to your path to make pdflatex easily executable. You can add this line to your path: /Library/TeX/texbin. For Windows, my understanding is that a pdf conversion tool should come with most {\sf TeX} distributions. Alternatively, you can do all rendering in {\sf Overleaf}.

Second, you will need to link your local project folder (such as an {\sf R} project) to a {\sf git} repository. It makes sense to use a remote repository, so your work will be backed up and you can easily access it from other computers or share the repository with collaborators. At the time of writing, both {\sf Github} and {\sf Bitbucket} offered free private repositories. Third, if you use {\sf Overleaf}, then you will need to link your {\sf Overleaf} project to the aforementioned repository. 

For information on how to do all this, see the excellent tutorial created by Jessica Rick (\url{https://github.com/jessicarick/resources})

\subsubsection{How {\sf Make} works}

When one runs {\sf Make} it looks for a ``Makefile'' that includes instructions for how {\sf Make} should run. Specifically, a Makefile will include dependency structures and commands that define the build order for a project. For example, a Makefile could say script X depends on script Y and data Z, if Z changes then re-execute Y and then re-execute X. 

Inside a Makefile are a series of commands following this structure: 

\begin{center}
target: dependencies\\
\indent instructions
\end{center}

\noindent where ``target'' is a file in your project, such as a manuscript; ``dependencies'' are the scripts, data, etc. that the target depends upon (e.g. the figures and results that go into your manuscript); and, ``instructions'' are things that must be done to the dependencies in order to successfully get the output that the target requires to build correctly. 

Importantly, the instructions line must be indented with a tab, if you use spaces your Makefile will fail with an error.

For our tutorial, we will use a project with the following dependency structure: a \LaTeX\ manuscript that relies on a figure and some results from two {\sf R} scripts, which, in turn, rely on a single csv data file. We also want to use {\sf git} as a version control system for all the files in our project, and sync any local changes with a remote repository. In particular, we want to send our \LaTeX\ manuscript to a remote repo so that we can download it into {\sf Overleaf} and use all its handy features. \emph{Note, if changes are made in Overleaf then those changes must be ``pulled'' to the local repository to see them. It is simplest to start every work session by pulling changes from the remote repository, so you stay synced up. If you want to take your work in a new direction, then learn about making ``branches'' in {\sf git}. If you don't understand what I am talking about read Jessi's tutorial, linked above.}   

The repository at \url{https://github.com/JHarrisonEcoEvo/Reproducible_workflow_tutorial} emulates the project structure verbally defined above. Clone this repository and use it to practice! The file ``main.tex'' will be our example manuscript. This also functions as a simple \LaTeX\ template that one could use for scientific manuscripts. There are some tips and tricks in this file, and a justification for using \LaTeX, so it is worth a scan. There are also two \.R scripts in the example folder, a \.sty file defining our \LaTeX\ style, a directory holding the data, a directory holding the results, and of course a Makefile. 

\subsubsection{Example code}

\verbatiminput{linearModel.R}

\end{document}
